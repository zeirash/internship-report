\renewcommand{\baselinestretch}{1.5}\normalsize

\begingroup
\begin{center}
\fontsize{16pt}{12pt}\selectfont
\textbf{PREFACE}
\end{center}
\endgroup
\justify
\tab Segala puji dan syukur penulis ucapkan atas kehadirat Tuhan Yang Maha Esa. Atas berkat karunia dan rahmat-Nya penulis dianugerahi kesehatan jasmani dan rohani serta nikmat ilmu pengetahuan sehingga skripsi dengan judul “Analisa dan Perancangan Aplikasi Manajemen Servis pada Bengkel JDM Project Berbasis Socket Programming” dapat diselesaikan tepat pada waktunya.

\tab Skripsi ini adalah buah dari perjuangan dan semangat penulis. Dalam penyajiannya, penulis menyadari masih terdapat kelemahan-kelemahan pada skripsi ini. Hal ini dikarenakan keterbatasan kemampuan dan pengetahuan yang dimiliki penulis. Namun berkat bimbingan dan arahan dari berbagai pihak, baik secara langsung atau tidak langsung, penulis dapat mengatasi hambatan-hambatan dalam menyusun skripsi ini hingga selesai. Untuk itu dalam kesempatan ini penulis ingin menyampaikan terima kasih kepada pihak-pihak yang telah membantu, khususnya kepada:

\begin{enumerate}
\itemsep0em
\item Prof. Dr. Ir. Harjanto Prabowo, M.M., as the Rector of Bina Nusantara University.
\item Mr. Fredy Purnomo, S.Kom., M.Kom., as Dean of School of Computer Science Bina Nusantara University.
\item Bapak Derwin Suhartono S.Kom., M.T.I., as Head of Computer Science Program Bina Nusantara University.
\item Ibu Yen Lina Prasetio, S.Kom., MCompSc as Deputy Head of School of Compter Science - Academic and Operation
\item Ms. Evawaty Tanuar, S.Kom., M.InfoTech as Deputy Head of School of Computer Science - Student and Alumni
\item Bapak Yohan Muliono, S.Kom., M.TI. selaku Dosen pembimbing.
\item Ms. Pauline Lim, gelar as
\item Para dosen di Universitas Bina Nusantara yang telah memberikan bekal ilmu yang menjadi modal dasar dalam menyusun skripsi ini.
\item Tidak lupa penulis ucapkan terima kasih yang sedalam-dalamnya untuk keluarga terutama kedua orang tua, teman-teman dan semua pihak yang tidak dapat disebutkan satu persatu, atas semangat dan dorongan yang telah diberikan kepada penulis sehingga skripsi ini dapat diselesaikan.
\end{enumerate}
Penulis sangat bersyukur apabila skripsi ini dapat berguna dan menambah pengetahuan bagi pembaca. Penulis menyadari bahwa skripsi ini masih memiliki kekurangan. Oleh karena itu, saran dan kritik yang membangun dari rekan-rekan pembaca sangat dibutuhkan untuk mengembangkan penelitian ini lebih jauh lagi.

Akhir kata, penulis mengucapkan permohonan maaf sebesar-besarnya atas segala kekurangan dalam penulisan skripsi ini.

\begin{flushright}
Jakarta, January 2018
\end{flushright}